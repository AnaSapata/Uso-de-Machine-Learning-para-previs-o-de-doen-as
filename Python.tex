\documentclass{article}

\title{Python}

\usepackage{listingsutf8}
\lstset{language=Python}
\lstset{
  literate = 
  {ã}{{\~a}}1
  {ç}{{\c c}}1
  {á}{{\'a}}1
  }

\setlength{\topmargin}{-0.4in}
 \setlength{\topskip}{0.3in}        % between header and text
 \setlength{\textheight}{9.5in}     % height of main text
 \setlength{\textwidth}{6in}        % width of text
 \setlength{\oddsidemargin}{0.1in}  % odd page left margin
 \setlength{\evensidemargin}{0.1in} % even page left margin


\begin{document}
  \maketitle
  \section{Conceitos Básicos}
  
  \begin{itemize}
  \item Defini\c c\~ao de variaveis:
  
  \begin{lstlisting}
  <nome_variavel> = <valor>
  \end{lstlisting}
  
  Exemplos:
  
  \begin{lstlisting}
  string = "Olá!"
  inteiro = 4
  decimal = 3.14
  \end{lstlisting}
  
  \item Arrays
  
  Os arrays s\~ao criados com parentesis retos [], estando os elementos do mesmo dentro destes separados por v\'irgulas.
  
  \begin{lstlisting}
  <nome_array> = [elem1, elem2, ..., elemN]
  \end{lstlisting}
  
  Exemplo:
  
  \begin{lstlisting}
  array = [1, 2, 3, "a dog"]
  \end{lstlisting}
  
  \begin{itemize}
  \item Adicionar elementos
  
  Para adicionar elementos a um array utiliza-se o m\'etodo append, seguido do elemento que se pretende adicionar.
  
  Exemplo:
  
  \begin{lstlisting}
  array.append(5)
  print(array)
    [1, 2, 3, "a dog", 5]
  \end{lstlisting}
  
  \item Selecionar elementos de um array
  
  Para se selecionar elementos de um array, utiliza-se o \'indice do elemento que pretendemos selecionar dentro de parentesis retos, a seguir ao nome do array.
  
  Exemplo:
  
  \begin{lstlisting}
  #Selecionar a string "a dog" do array, sendo o seu indice 3
  print(array[3])
  	a dog
  	
  #selecionar o numero 3 do array, sendo o seu indice 2, e atribuir o mesmo à variavel num
  num = array[2]
  print(num)
  	3
  \end{lstlisting}
  
  \item Eliminar elementos de um array
  
  Para eliminar elementos de um array é utilizada a fun\c c\~ao del sendo dado como o seu parametro o nome do array seguido do indice do elemento que se pretende eliminar entre parentesis retos [].
  
  Exemplo:
  
  \begin{lstlisting}
  #queremos apenas que o array contenha numeros, pelo que se irá eliminar 
  #a string "a dog" contido no indice 3
  del(array[3])
  print(array)
  	[1, 2, 3, 5]
  \end{lstlisting}
  
  \item Adicionar elementos num indice especifico
  
  Para se adicionar elementos ao array, num indice especifico \'e utilizado o m\'etodo insert aplicado ao array seguido do indice onde se pretende adicionar e o elemento que se aprende adicionar no mesmo.
  
  \begin{lstlisting}
  <nome_array>.insert(<indice>, <elemento>)
  \end{lstlisting}
  
  Exemplo:
  
  \begin{lstlisting}
  #inserir novamente a string "a dog" no array no indice 3, 
  #onde estava anteriormente
  array.insert(3, "a dog")
  print(array)
  	[1, 2, 3, "a dog", 5]
  \end{lstlisting}
  
  \item Selecionar um subconjunto do array
  
  \begin{lstlisting}
  <nome_array>[indice_inicial:indice_final]
  \end{lstlisting}
  
  Exemplo:
  
  \begin{lstlisting}
  array[2:4]
  	[3, "a dog"]
  \end{lstlisting}
  
  \end{itemize}  
  
  \item Ciclo for
  \begin{lstlisting}
  for <condição>:
	<desenvolvimento do que fará quando se verifica a condição>
  \end{lstlisting}  
  
  Exemplo:
  \begin{lstlisting}
  for x in range (0,3): 
  	print "We're on time" + x 	
  \end{lstlisting}
  Quando x pertence ao intervalo [0;3[ (range(a,b) cria o intervalo/sequencia [a.b[) escreve/imprime a frase We're on time x, substituindo x pelo respetivo valor
  
  \item Ciclo while  
  \begin{lstlisting}
  for <condição>:
	<desenvolvimento do que fará quando se verifica a condição>
  \end{lstlisting} 
  
  Exemplo:
  \begin{lstlisting}
  x = 1  
  while x<3: 
  	print "We're on time" + x
  	x += 1
  \end{lstlisting}
  Enquanto x for inferior a 3, imprime We're on time x, substituindo x pelo respetivo valor, incrementando de seguida 1 ao mesmo
  
  \item Condicionais if/else
  \begin{lstlisting}
  if <condição>:
  	<o que faz se condição verdade>
  else:
  	<o que faz se condição falsa>
  \end{lstlisting}
  
  Exemplo:
  
  \begin{lstlisting}
  num = 3
  if num >= 0:
  	print("Positivo ou zero")
  else:
  	print("Negativo")
  \end{lstlisting}
  Se a variavel num for maior ou igual que zero imprime a frase "Positivo ou zero", caso contr\'ario imprime a frase "Negativo"
  
  \item Defini\c c\~ao de fun\c c\~oes:
  
  \begin{lstlisting}
  def <nome> (<args>):
  	<corpo da função
  	termina com return>
  \end{lstlisting}
  
  Exemplo:
  \begin{lstlisting}
  def soma (a, b):
	return a + b
  \end{lstlisting}  
  
\end{itemize}

\section{Numpy}

\section{Pandas}

\section{Matplotlib}

\section{scikit-learn}
    
\end{document}
